\documentclass[11p]{article}
% Packages
\usepackage{amsmath}
\usepackage{graphicx}
\usepackage[swedish]{babel}
\usepackage[
    backend=biber,
    style=authoryear-ibid,
    sorting=ynt
]{biblatex}
\usepackage[utf8]{inputenc}
\usepackage[T1]{fontenc}
%Källor
\addbibresource{mall.bib}
\graphicspath{ {./images/} }

\title{PMmall \\ \small Fysik 1}
\author{Gabriel Nilsson Högberg}
\date{\today}

\begin{document}

    \begin{titlepage}
        \begin{center}
            \vspace*{1cm}

            \Huge
            \textbf{Avfall som energiförsörjning}

            \vspace{0.5cm}
            \LARGE
            Biobränsle

            \vspace{1.5cm}

            \textbf{Gabriel Nilsson Högberg}

            \vfill

            Ett PM om energiförsörjning \\
            Fysik 1

            \vspace{0.8cm}

            \includegraphics[width=0.4\textwidth]{../images/NTI Gymnasiet_Symbol_print_svart.png}

            \Large
            Teknikprogrammet\\
            NTI Gymnasiet\\
            Umeå\\
            \today

        \end{center}
    \end{titlepage}
% Om arbetet är långt har det en innehållsförteckning, annars kan den utelämnas
    \tableofcontents
    \newpage

    \section{Inledning}
    Ämnet handlar om avfall som energiförsörjning, alltså sopor som bränns för att bilda värme och el. Det som är viktigt att veta är hur detta funkar och hur detta påverkar miljön och samhället både lokalt och globalt.

    \subsection{frågeställningar}
    De frågorna som ska besvaras är följande:
    \begin{enumerate}
        \item Hur fungerar avfallsförbränning?
        \item Vilken sorts miljöpåverkan har avfallsförbränning lokalt och globalt?
        \item Hur påverkar avfallsförbränning samhället lokalt och globalt?
    \end{enumerate}

    \section{Resultat}
    \subsection{Avfallsförbränning, så fungerar det}
    Enligt \textcite{SoporNu} fungerar avfallsförbränning genom att man har avfallseldade kraftvärmeverk som ska kunna producera både värme och el genom att bränna sopor. I själva kraftvärmeverket överförs värme av rökgaser till vatten som bildar ånga och som i sin tur driver en turbin för elproduktion. Sedan förs den resterande energin till fjärrvärmenätet som värmer hushåll i Sverige.
    \subsection{Lokala och globala miljöpåverkningar av avfallsförbränning}
    Vid förbränningsprocessen av avfall frigörs framförallt koldioxid och vatten och enligt \textcite{AvfallSverige} är det ett hygieniskt och miljömässigt bra sätt eftersom vi har otroligt bra reningstekniker och förbränningsförhållanden samtidigt som vi har bra kontroll av att inte släppa ut föroreningar av avfallsförbränningen. Idag kan man också nästan helt få rökgaserna som släpps ut att bestå av 99,9 procent ämnen som redan finns i luften, det vill säga kväve, vattenånga, koldioxid och syre.

    \newline

    Det finns däremot stora negativa påverkningar av miljön. Enligt \textcite{naturvardsverket} är bränderna svårsläckta och kan pågå i flera veckor och dem kan fortfarande släppa ut farliga gaser även fast vi har bra reningstekniker. Dessutom där soporna hamnar kallas för en deponi (soptip) och dessa samlar stora mängder föroreningar och miljögifter som sedan förs vidare av lakvatten, vatten från regn som därefter för dessa dåliga ämnen vidare. Själva deponierna ger upphov till utsläpp av ämnet metangas som i sig är en av dem värsta växthusgaserna. Metangas har nästan 25 gånger större påverkan än koldioxid och detta är ett av Europas största miljöproblem enligt \textcite{Vattenfall}.
    \subsection{Hur avfallsförbränning påverka samhället lokalt och globalt}

    \section{Slutsatser}
    Här kan du dra slutsatser eller sammanfatta ditt resultat

% Mer saker som du kan ha nytta av.

    \section{Referenser}
    Referenser i text kan skrivas på två sätt: Enligt \textcite{Jens} kan man använde två typer av referenser, inbäddade i texten eller efter ett fakta \parencite{Fraenkel}. Ett till test för att se hur det ser ut \parencite[sid 55]{fermi}.

    \section{Annat som kan vara bra att veta}
    Om du vill ha kodstil och få med alla tecken kan du använda verbatim. då kan du skriva \verb|abcd!"#| utan problem...

    Citat skrivs mellan de konstiga symbolerna \verb|``| och \verb|''| för att de ska se bra ut ``se bra ut!''.
    \subsection{En underrubrik}
    \subsubsection{En underunderrubrik}
    \subsection{Ekvationer}
    Det är lätt att skriva matematik i \LaTeX

    \begin{equation}
        F = G \frac{M m}{r^2}
        \label{grav}
    \end{equation}

    Ekvation (\ref{grav}) känner ni igen...

    \subsection{figurer}
    Bilder placeras enklast på detta sätt. Placeringen bestämmer \LaTeX och vi kan bara föreslå (h)är, (t)opp eller (b)otten. Ett utropstecken före tvingar lite mer men inte absolut. I bild \ref{varg} visas en varg
    \begin{figure}[!h]
        \includegraphics[width=0.8\textwidth]{../images/accelerationTime.png}
        \caption{Acceleration-tid diagram. Källa: Impuls Fysik 1}
        \label{varg}
    \end{figure}
    \printbibliography

\end{document}
